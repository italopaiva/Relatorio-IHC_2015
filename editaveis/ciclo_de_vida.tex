\chapter{Ciclo de vida}

  A Engenharia de Usabilidade (também denominada Design Centrado no Usuário) trata da criação de sistemas melhores através da 
  compreensão de quem são usuários finais e do envolvimento de usuários nos requisitos, no design de interface do usuário e nos
  esforços de teste. O modelo que é apoiado na engenharia de software é baseado em três tarefas essenciais: Análise de requisitos
  e Projeto/teste/desenvolvimento e Instalação.
  
  O ciclo de vida estrela é uma proposta que não especifica um ordenamento das atividades, mas sua flexibilidade exige que uma
  avaliação sempre seja feita antes de iniciar uma nova atividade. Este modelo inclui a implementação e a análise de tarefas.
  As  outras atividades vão de encontro com as atividades básicas dos projeto de interação: necessidade e requisitos, versões 
  alternativas (projeto conceitual) e protótipos e avaliação.
  
  O processo de design seguido foi o modelo Estrela, que envolve as atividades de Implementação, Análise das tarefas, Prototipação,
  Projeto conceitual e Requisitos, onde é possível começar o desenvolvimento por qualquer uma das atividades e todas as atividades
  convergem para a avaliação.
  
  Primeiramente foi feito o projeto conceitual da solução e levantado requisitos iniciais. Logo após, foi iniciada a atividade
  de prototipação com a confecção do storyboad e o protótipo de papel, onde foram levantados mais requisitos e alguns refinados.