\chapter{Planejamento da avaliação}

  Faz parte do escopo da avaliação, a análise, sob a ótica de IHC, de todos os requisitos funcionais e os requisitos não-funcionais
  RNF1 e RNF3, listados acima, analisando os princípios e metas de usabilidade desejados para o produto e as metas decorrentes
  da experiência do usuário.
  
  A avaliação tem por objetivos avaliar a aceitação da tecnologia pelo usuário, validar ideias e designs alternativos e buscar
  por problemas na interface e na interação.
  
  A análise dos \textit{storyboards} e do protótipo de papel fornecem insumos para identificar problemas na interação e na interface,
  sendo a primeira avaliação a ser feita no projeto, com caráter inspecionista.
  
  Após a construção de um protótipo de alta fidelidade ou da implementação do sistema completo, pretende-se fazer uma análise
  de laboratório com um grupo de usuários (a definir) a partir de observação de uso. Também será feita uma análise investigativa
  por meio de questionários e entrevistas.

  \section{Estudos dos questionários}
    
    Os seguintes questionários foram analisados para a definição dos que seriam utilizados no projeto.
    
    \subsection{\textit{Software Usability Measurement Inventory} - SUMI}
      
      SUMI é um rigoroso questionário que tem a finalidade de medir a qualidade de um software do ponto de vista do usuário.
      Ele é composto por 50 questões e possui três níveis de resposta: Concordo, Indeciso e Não Concordo. O SUMI seria utilizado
      para avaliar a qualidade dos requisitos do \textit{MyPush} pelos usuários. Essas avaliações possibilitariam que o aplicativo fosse
      melhorado a cada versão.
      
    \subsection{\textit{Website Analysis MeasureMent Inventory} - WAMMI}
    
      O \textit{Website Analysis MeasureMent Inventory} (WAMMI) é um serviço analítico para a \textit{web} que mede e analisa a experiência do
      usuário em um \textit{website}, baseado na reação dos visitantes. Essa reação dos visitantes é medida com base na comparação das
      expectativas do usuário com o que realmente foi encontrado no \textit{site}, por meio de um questionário de 20 questões que pode
      ser expandido com algumas questões adicionais.
      
      As questões do WAMMI permitem analisar cinco fatores de usabilidade, Atratividade, Controlabilidade, Eficiência, Utilidade
      e Capacidade de aprendizado por meio de um padrão de resposta gradativo de cinco escalas que varia de “Concordo fortemente”
      a “Discordo fortemente”.
      
      Os serviços de análise das questões fornecidos pelo WAMMI são únicos, devido a todo arcabouço matemático e científico
      envolvido na sua construção, e, apesar do WAMMI ser voltado para \textit{websites}, para poder utilizá-lo no projeto, as perguntas
      podem ser adaptadas ao contexto do projeto para fornecer uma avaliação das metas de usabilidade Atratividade,
      Controlabilidade, Eficiência, Utilidade e Capacidade de aprendizado da aplicação.
    
    \subsection{\textit{Questionnaire for User Interface Satisfaction} - QUIS}
    
      O QUIS é um tipo de questionário que avalia a satisfação do usuário em relação a usabilidade do produto quanto à sua
      padronização, a fim de obter informações de forma precisa em relação a reação dos usuários aos seus novos produtos. 
      
      O pacote QUIS é composto por questões que podem ser avaliadas em uma escala de 0 a 9 e este pacote é possui um documento
      de texto composto por todas as seções do questionário que podem ser editadas de acordo com a necessidade de avaliação do 
      \textit{software}, possui uma versão única do questionário aplicado em HTML e possui também uma seleção dos trabalhos relevantes
      que detalham a validação do QUIS e alguns de seus usos. 
      
      O aplicativo \textit{MyPush} seguirá alguns padrões de interface para atender melhor a satisfação dos usuários na utilização do
      aplicativo. Esse tipo de questionário pode ser aplicado para garantir que a usabilidade seja levada em conta no processo de
      criação de interfaces do aplicativo e para verificar se o aplicativo está de conforme com as expectativas do usuário, 
      avaliando o seu grau de satisfação em relação as funcionalidades do \textit{MyPush}.
      
    \subsection{\textit{ErgoList}}
    
      O questionário ErgoList é um serviço disponibilizado via internet composto de uma base de conhecimento em ergonomia,
      que inspeciona, através de um \textit{checklist}, interfaces homem-computador. Nesse ambiente, o especialista avalia a interface 
      de uma aplicação usando o \textit{checklist} disponibilizado pelo LabIUtil. Tem uma base de questões bastante completa, cobrindo
      vários aspectos da usabilidade do \textit{software}. Por ser um \textit{checklist}, deve ser usado por um especialista com conhecimento em
      ergonomia (ERGOLIST, 2008).
      
      Este questionário leva em consideração os dezoito critérios ergonômicos de Bastien e Scapin (1993). São dezoito \textit{checklists}
      disponíveis no Ergolist, para cada um desses critérios que determina a ergonomia de uma interface homem-computador,
      totalizando 194 questões.
      
      O Ergolist é um questionário bem completo de tal modo que possui uma excelente precisão quando usado por um especialista,
      sendo muito útil para a medição.	

      Como o \textit{MyPush} é um aplicativo que pretende seguir os padrões de interface ditados pela indústria, é de vital
      importância para o seu desenvolvimento que estudos rígidos como o ErgoList sejam aplicados para que possa manter o 
      padrão desejado e o mesmo não fique aquém do esperado.
    
    \subsection{\textit{After Scenario Questionnaire} - ASQ}
    
      O questionário ASQ é um questionário feito para ser usado, como o próprio nome diz, imediatamente após cada cenário de uso
      completo, que permite avaliar a satisfação do usuário durante a participação de um estudo de usabilidade baseado em cenários,
      onde um cenário é um conjunto de tarefas relacionadas \cite{lewis91}.
      
      Consiste em apenas três itens que possuem padrão de resposta gradativo de sete escalas, que varia de “Concordo fortemente” 
      a “Discordo fortemente” e também conta com a opção “Não aplicável” fora da escala. A pontuação do questionário pode ser
      obtida pela média aritmética dos pontos de cada questão.
      
      Os aspectos abordados nos itens do ASQ são: facilidade de completar as tarefas, tempo necessário para completar as tarefas 
      e satisfação com informações de suporte durante as tarefas, que possibilita visualizar a percepção do usuário a respeito da
      usabilidade do sistema \cite{lewis91}.
      
      O questionário ASQ mostra se um recurso de avaliação rápido e eficaz para avaliar imediatamente a satisfação do usuário
      frente a um cenário de uso, evidenciando sua passividade de uso no projeto.
    
    \subsection{\textit{Post Study System Usability Questionnaire} - PSSUQ}
      
      O questionário PSSUQ é um instrumento que também permite avaliar a satisfação percebida pelo usuário ao utilizar um
      sistema \cite{lewis02}. Diferentemente do ASQ, o PSSUQ deve ser aplicado após completado um conjunto definido de cenários,
      para avaliar de forma generalizada o sistema (um conjunto de cenários).
      
      Assim como o ASQ, o PSSUQ também possui um padrão gradativo de resposta de sete escalas que varia de “Concordo fortemente”
      a “Discordo fortemente” e possui a opção “Não aplicável” fora da escala, porém conta com dezenove itens para a avaliação,
      onde cada item tem a possibilidade de receber um comentário do usuário. Os itens do PSSUQ permite avaliar o sistema em quatro
      aspectos, fornecendo medidas para cada um deles. São eles:
      
      \begin{itemize}
       \item \textit{SysUse} - Avalia a usabilidade do sistema (\textit{System Usefulness}).
	  \subitem Questões 1-8;
      
       \item \textit{InfoQual} - Avalia a qualidade da informação fornecida pelo sistema. (\textit{Information Quality})
	  \subitem Questões 9-15;
	  
       \item \textit{InterQual} - Avalia a qualidade da interface do sistema (\textit{Interface Quality}).
	  \subitem Questões 16-18;
       
       \item \textit{Overall} - Avalia a satisfação geral do usuário com o sistema.
	  \subitem Questões 1-19.

      \end{itemize}
      
      Para obter a pontuação de qualquer área, basta calcular a média aritmética das pontuações das questões relacionadas à área.
      
      Um ponto interessante deste questionário é que, devido a utilização de técnicas psicométricas, a não observância de
      algumas questões não impacta significativamente no resultado obtido, ou seja, um questionário incompleto possui, 
      tecnicamente, a mesma confiabilidade de um questionário completo \cite{lewis02}. Outro ponto interessante é que o
      resultado do PSSUQ não possui variações significativas em detrimento ao sexo do usuário \cite{lewis02}.
      
      Com o que foi exposto, percebe-se que o PSSUQ é um questionário poderoso que permite uma avaliação confiável da
      satisfação do usuário em relação ao sistema. Como o PSSUQ deve ser utilizado após um conjunto de cenários ter sido
      completado, o seu uso concomitante com o ASQ se mostra uma prática promissora para a avaliação, pois, enquanto o ASQ
      retorna um feedback a cada cenário realizado, o PSSUQ permite um feedback geral do sistema que está sendo analisado.

