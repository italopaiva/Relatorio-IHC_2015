\chapter{Protótipos}

  Essa seção apresenta os protótipos realizados ao longo do projeto.

  \section{\textit{Storyboard}}
    
    Para o entendimento do contexto de uso da aplicação, foi feito um rápido esboço de um \textit{storyboard} no papel para ilustrar
    um cenário de uso. Logo após foi feita a construção de um \textit{storyboard} mais elaborado utilizando a ferramenta
    \textit{Bitstrips}, para melhor representação do cenário de uso. Os apêndices A e B apresentam, respectivamente,
    o esboço inicial do \textit{storyboard} e o \textit{storyboard} feito na ferramenta.
    
    
    Ambos \textit{storyboards} descrevem um aluno em sua instituição de ensino fazendo suas atividades corriqueiras quando
    repara um calendário na parede e percebe que iria ter um feriado no dia 05 de abril de 2015 e resolve que deseja fazer algo no
    feriado. Para tal, ele utiliza o \textit{MyPush} por causa da sua facilidade de programar notificações em relação ao tempo, para programar
    notificações sobre eventos no dia do feriado. Quando ele recebe a notificação sobre os eventos do dia, ele escolhe ir ao \textit{stand up}
    no teatro para curtir o feriado com muito humor.
    
  \section{Protótipo de papel}

    Inicialmente foi construído um protótipo de papel para ilustrar os requisitos funcionais que a aplicação teria, e para permitir 
    uma rápida avaliação do sistema em busca de novos requisitos.
    
    Os apêndices C e D ilustram o protótipo de papel construído (em diferentes versões) para implementar conceitualmente os
    requisitos funcionais descritos. O planejamento e resultados da execução das avaliações do protótipo de papel se encontram 
    nas seções 7 e 8, respectivamente.
    
    
  \section{Protótipo de alta fidelidade}
  
    Após a avaliação, e consequente evolução, do protótipo de papel, as funcionalidades descritas neste foram prototipadas na ferramenta 
    \textit{JustinMind} para avaliar os quesitos de usabilidade da aplicação.
    
    Os apêndices E, F e G ilustram as telas do protótipo modelado (em diferentes versões). O planejamento e resultados da execução
    das avaliações feitos para o protótipo de alta fidelidade também se encontram nas seções 7 e 8.