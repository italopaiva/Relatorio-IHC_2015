  Logo abaixo são apresentados os resultados da segunda iteração de avaliação do protótipo de alta fidelidade,
  seguindo o planejamento realizado.
  
  \begin{itemize}
    \item \textbf{Descrição e metodologia do roteiro da avaliação}
    
    \subitem O objetivo da avaliação era analisar qual a satisfação obtida pelo usuário ao utilizar o protótipo, 
    considerando as seguintes metas:
    
    \begin{itemize}
      \item Utilidade;
	\subitem Avaliada pela dimensão \textit{SysUse} do questionário PSSUQ.
	
      \item Eficácia e eficiência;
	\subitem Avaliada pelas dimensões \textit{SysUse} e \textit{InfoQual} do questionário PSSUQ.
	
      \item Interface esteticamente agradável.
	\subitem Avaliada pela dimensão \textit{InterQual} do questionário PSSUQ.
    \end{itemize}
    
    \item \textbf{Comportamento dos usuários}
    
      \subitem 
    
    \item \textbf{Resumo das entrevistas}
    
      \subitem 
	    
    \item \textbf{Problemas de usabilidade identificados}
    
      \subitem 
          
    \item \textbf{Paradas críticas}
    
      \subitem Não houveram paradas críticas nas avaliações realizadas.
    
    \item \textbf{Plano de correção}
    
      \subitem 
    
  \end{itemize}
  
  Na figura \ref{asqalta_2} se encontram as respostas ao questionário ASQ pelos cinco usuários avaliados.
  
%   \begin{figure}[!htb]
%   \centering
%   \includegraphics[scale=0.6]{}
%   \caption{Resposta dos usuários ao questionário ASQ na segunda avaliação do protótipo de alta fidelidade}
%   \label{asqalta_2}
%   \end{figure}
%   
%   Na figura \ref{pssuqalta_2} se encontram as respostas ao questionário PSSUQ pelos cinco usuários avaliados.
%   
%   \begin{figure}[!htb]
%   \centering
%   \includegraphics[scale=0.6]{}
%   \caption{Resposta dos usuários ao questionário PSSUQ na segunda avaliação do protótipo de alta fidelidade}
%   \label{pssuqalta_2}
%   \end{figure}  