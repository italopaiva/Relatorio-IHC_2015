  Logo abaixo são apresentados os resultados da segunda iteração de avaliação do protótipo de alta fidelidade,
  seguindo o planejamento realizado.
  
  \begin{itemize}
    \item \textbf{Descrição e metodologia do roteiro da avaliação}
    
    \subitem O objetivo dessa avaliação é analisar qual a satisfação obtida pelo usuário ao utilizar o protótipo 
    considerando as seguintes metas, além dos princípios de usabilidade de Nielsen apresentados (vide seção 4.3):
    
    \begin{itemize}
      \item Utilidade
      \item Eficácia
      \item Eficiência
    \end{itemize}
    
    \item \textbf{Comportamento dos usuários}
    
      \subitem 
    
    \item \textbf{Resumo das entrevistas}
    
      \subitem 
	    
    \item \textbf{Problemas de usabilidade identificados}
    
      \subitem 
          
    \item \textbf{Paradas críticas}
    
      \subitem Não houveram paradas críticas nas avaliações realizadas.
    
    \item \textbf{Plano de correção}
    
      \subitem 
    
  \end{itemize}
  
  Na figura \ref{asqalta_2} se encontram as respostas ao questionário ASQ pelos cinco usuários avaliados.
  
  \begin{figure}[!htb]
  \centering
  \includegraphics[scale=0.6]{}
  \caption{Resposta dos usuários ao questionário ASQ na segunda avaliação do protótipo de alta fidelidade}
  \label{asqalta_2}
  \end{figure}
  
  Na figura \ref{pssuqalta_2} se encontram as respostas ao questionário PSSUQ pelos cinco usuários avaliados.
  
  \begin{figure}[!htb]
  \centering
  \includegraphics[scale=0.6]{}
  \caption{Resposta dos usuários ao questionário PSSUQ na segunda avaliação do protótipo de alta fidelidade}
  \label{pssuqalta_2}
  \end{figure}  