\begin{anexosenv}

\partanexos

\chapter{\textit{Storyboard} feito à mão.}

 \begin{figure}[!htbp]
    \centering
    \includegraphics[scale=0.2]{editaveis/figuras/storyboard_papel}
    \caption{\textit{Storyboard} feito à mão.}
    \label{storyboard_papel}
  \end{figure}

\chapter{\textit{Storyboard} feito com a utilização da ferramenta \textit{Bitstrips}.}

 \begin{figure}[!htbp]
    \centering
    \includegraphics[scale=0.5]{editaveis/figuras/storyboard_ferramenta}
    \caption{\textit{Storyboard} feito com a utilização da ferramenta \textit{Bitstrips}.}
    \label{storyboard_ferramenta}
  \end{figure}

\chapter{Protótipo de papel - Versão 1.0}

  \section*{Protótipo de papel que ilustra a página inicial do sistema.}
  
    \begin{figure}[!htbp]
      \centering
      \includegraphics[scale=0.32]{editaveis/figuras/prototipo_papel_v1/pagina_inicial}
      \caption{Protótipo de papel que ilustra a página inicial do sistema.}
      \label{pagina_inicial_v1}
    \end{figure}
  
    \pagebreak
    \section*{Protótipo de papel que ilustra a funcionalidade procurar por tema.}
    
      \begin{figure}[!htbp]
	\centering
	\includegraphics[scale=0.32, angle=-90]{editaveis/figuras/prototipo_papel_v1/procurar_por_temas}
	\caption{Protótipo de papel que ilustra a funcionalidade procurar por tema.}
	\label{procurar_por_temas_v1}
      \end{figure}
    
    \pagebreak
    \section*{Protótipo de papel que ilustra a funcionalidade procurar por tema quando um tema é adicionado.}
    
      \begin{figure}[!htbp]
	\centering
	\includegraphics[scale=0.32]{editaveis/figuras/prototipo_papel_v1/tema_adicionado}
	\caption{Protótipo de papel que ilustra a funcionalidade procurar por tema quando um tema é adicionado.}
	\label{tema_adicionado_v1}
      \end{figure}
	
    \pagebreak
    \section*{Protótipo de papel que ilustra a funcionalidade de escolher a data e hora para a notificação.}
    
      \begin{figure}[!htbp]
	\centering
	\includegraphics[scale=0.32, angle=-90]{editaveis/figuras/prototipo_papel_v1/quando_notificar}
	\caption{Protótipo de papel que ilustra a funcionalidade de escolher a data e hora para a notificação.}
	\label{quando_notificar_v1}
      \end{figure}
      
    \pagebreak
    \section*{Protótipo de papel que ilustra a funcionalidade de escolher a data por um calendário.}
    
      \begin{figure}[!htbp]
	\centering
	\includegraphics[scale=0.32, angle=-90]{editaveis/figuras/prototipo_papel_v1/quando_notificar_data}
	\caption{Protótipo de papel que ilustra a funcionalidade de escolher a data por um calendário.}
	\label{quando_notificar_data_v1}
      \end{figure}
    
    \pagebreak
    \section*{Protótipo de papel que ilustra a funcionalidade de escolher a hora por um \textit{dropdown}.}
    
      \begin{figure}[!htbp]
	\centering
	\includegraphics[scale=0.32, angle=-90]{editaveis/figuras/prototipo_papel_v1/quando_notificar_horario}
	\caption{Protótipo de papel que ilustra a funcionalidade de escolher a hora por um \textit{dropdown}.}
	\label{quando_notificar_horario_v1}
      \end{figure}
      
    \pagebreak
    \section*{Protótipo de papel que ilustra a mensagem de notificação salva.}
    
      \begin{figure}[!htbp]
	\centering
	\includegraphics[scale=0.32, angle=-90]{editaveis/figuras/prototipo_papel_v1/notificacao_salva}
	\caption{Protótipo de papel que ilustra a mensagem de notificação salva.}
	\label{notificacao_salva_v1}
      \end{figure}
    
    \pagebreak
    \section*{Protótipo de papel que ilustra a página de notificações agendadas.}
    
      \begin{figure}[!htbp]
	\centering
	\includegraphics[scale=0.32, angle=-90]{editaveis/figuras/prototipo_papel_v1/notificacoes_agendadas}
	\caption{Protótipo de papel que ilustra a página de notificações agendadas.}
	\label{notificacoes_agendadas_v1}
      \end{figure}
    
    \pagebreak
    \section*{Protótipo de papel que ilustra a mensagem de confirmação de exclusão de notificação.}
    
      \begin{figure}[!htbp]
	\centering
	\includegraphics[scale=0.32, angle=-90]{editaveis/figuras/prototipo_papel_v1/confirmacao_excluir_notificacao}
	\caption{Protótipo de papel que ilustra a mensagem de confirmação de exclusão de notificação.}
	\label{confirmacao_excluir_notificacao_v1}
      \end{figure}
    
    \pagebreak
    \section*{Protótipo de papel que ilustra a página para edição de uma notificação.}
    
      \begin{figure}[!htbp]
	\centering
	\includegraphics[scale=0.32, angle=-90]{editaveis/figuras/prototipo_papel_v1/editar_notificacao}
	\caption{Protótipo de papel que ilustra a página para edição de uma notificação.}
	\label{editar_notificacao_v1}
      \end{figure}
    
    \pagebreak
    \section*{Protótipo de papel que ilustra um exemplo de notificação no celular.}
    
      \begin{figure}[!htbp]
	\centering
	\includegraphics[scale=0.32]{editaveis/figuras/prototipo_papel_v1/tela_bloqueio_notificacao}
	\caption{Protótipo de papel que ilustra um exemplo de notificação no celular.}
	\label{tela_bloqueio_notificacao_v1}
      \end{figure}
    
\chapter{Protótipo de papel - Versão 2.0}
  
  \section*{Protótipo da página inicial com as notificações agendadas}
  
    \begin{figure}[!htbp]
      \centering
      \includegraphics[scale=0.32, angle=-90]{editaveis/figuras/prototipo_papel_v2/pagina_inicial}
      \caption{Protótipo da página inicial com as notificações agendadas}
      \label{pagina_inicial_v2}
    \end{figure}
  
  \pagebreak
  \section*{Protótipo que ilustra a alteração na página de procurar por temas}
    
    \begin{figure}[!htbp]
      \centering
      \includegraphics[scale=0.32]{editaveis/figuras/prototipo_papel_v2/procurar_temas}
      \caption{Protótipo que ilustra a alteração na página de procurar por temas}
      \label{procurar_por_temas_v2}
    \end{figure}

  \pagebreak
  \section*{Protótipo que ilustra a alteração na página de procurar por temas}

    \begin{figure}[!htbp]
      \centering
      \includegraphics[scale=0.32]{editaveis/figuras/prototipo_papel_v2/procurar_temas}
      \caption{Protótipo que ilustra a alteração na página de procurar por temas}
      \label{procurar_por_temas_v2}
    \end{figure}
   
  \pagebreak
  \section*{Protótipo que ilustra a nova tela de quando notificar}

    \begin{figure}[!htbp]
      \centering
      \includegraphics[scale=0.32, angle=-90]{editaveis/figuras/prototipo_papel_v2/quando_notificar}
      \caption{Protótipo que ilustra a alteração na página de quando notificar}
      \label{quando_notificar_v2}
    \end{figure}
  
  \pagebreak
  \section*{Protótipo que ilustra a opção de escolher uma cidade na tela de quando notificar}

    \begin{figure}[!htbp]
      \centering
      \includegraphics[scale=0.32, angle=-90]{editaveis/figuras/prototipo_papel_v2/quando_notificar_cidade}
      \caption{Protótipo que ilustra a opção de escolher uma cidade na tela de quando notificar}
      \label{quando_notificar_cidade_v2}
    \end{figure}
    
  \pagebreak
  \section*{Protótipo que ilustra a informação sobre o horário}

    \begin{figure}[!htbp]
      \centering
      \includegraphics[scale=0.32, angle=-90]{editaveis/figuras/prototipo_papel_v2/quando_notificar_info_horario}
      \caption{Protótipo que ilustra a informação sobre o horário}
      \label{quando_notificar_info_horario_v2}
    \end{figure}
    
  \pagebreak
  \section*{Protótipo que ilustra a mensagem de confirmação para voltar para outra tela}

    \begin{figure}[!htbp]
      \centering
      \includegraphics[scale=0.32, angle=-90]{editaveis/figuras/prototipo_papel_v2/confirmacao_perda_dados}
      \caption{Protótipo que ilustra a mensagem de confirmação para voltar para outra tela}
      \label{confirmacao_perda_dados_v2}
    \end{figure}
    
  \pagebreak
  \section*{Protótipo que ilustra a nova tela para editar uma notificação}

    \begin{figure}[!htbp]
      \centering
      \includegraphics[scale=0.32, angle=-90]{editaveis/figuras/prototipo_papel_v2/editar_notificacao}
      \caption{Protótipo que ilustra a nova tela para editar uma notificação}
      \label{editar_notificacao_v2}
    \end{figure}
  
  \pagebreak
  \section*{Protótipo que ilustra a caixa para buscar mais temas}

    \begin{figure}[!htbp]
      \centering
      \includegraphics[scale=0.32, angle=-90]{editaveis/figuras/prototipo_papel_v2/editar_notificacao_mais_temas}
      \caption{Protótipo que ilustra a caixa para buscar mais temas}
      \label{editar_notificacao_mais_temas_v2}
    \end{figure}
  
  \pagebreak
  \section*{Protótipo que ilustra uma exclusão de uma notificação}

    \begin{figure}[!htbp]
      \centering
      \includegraphics[scale=0.32, angle=-90]{editaveis/figuras/prototipo_papel_v2/confirmacao_exclusao_notificacao}
      \caption{Protótipo que ilustra uma exclusão de uma notificação}
      \label{confirmacao_exclusao_notificacao_v2}
    \end{figure}
  
  \pagebreak
  \section*{Protótipo que ilustra a página de notificações agendas sem notificações}

    \begin{figure}[!htbp]
      \centering
      \includegraphics[scale=0.32, angle=-90]{editaveis/figuras/prototipo_papel_v2/confirmacao_exclusao_notificacao}
      \caption{Protótipo que ilustra a página de notificações agendas sem notificações (após a exclusão)}
      \label{confirmacao_exclusao_notificacao_v2}
    \end{figure}
  
  \chapter{Modelo do termo de consentimento}
  
          \begin{center}
      
       \textbf{Termo de Consentimento para Avaliação do protótipo do aplicativo MyPush}
       
      \end{center}
      
      Você foi convidado(a) pela equipe da disciplina Interação Humano Computador ofertada no curso de graduação de Engenharia de Software 
      pela Universidade de Brasília - UnB, para participar de um teste de avaliação da Interação Humano Computador do aplicativo MyPush, 
      um aplicativo projetado para o recebimento de notificações de eventos.
      
      Nesta avaliação, buscamos verificar a usabilidade e a decorrência da experiência do usuário interagindo com o \textit{Mypush}, durante a 
      realização de uma tarefa de anotação e observação.

      Por esta razão, solicitamos seu consentimento para a realização deste teste e, em seguida, que responda um breve questionário. 
      Para tanto, é importante que você tenha algumas informações: 
      
      1. Os dados coletados durante avaliação destinam-se apenas a atividades de pesquisa e de desenvolvimento.
      
      2. A equipe desta pesquisa tem o compromisso e responsabilidade de divulgar os resultados de suas pesquisas somente para fins acadêmicos 
      para que seja mantida a sua privacidade e anonimato. 
      
      3. Se o usuário aceita participar da avaliação do protótipo, o mesmo permite que seja gravado um video de curta duração no momento da avaliação. Esta gravação é exclusivamente para fins acadêmicos, e não será exposta a privacidade do usuário.
      
      4. O consentimento para a realização da avaliação é uma escolha livre, feita mediante a prestação de todos os esclarecimentos necessários 
      sobre a pesquisa.
      
      5. A realização da avaliação pode ser interrompida a qualquer momento, segundo a disponibilidade do participante. Neste caso, a equipe se 
      compromete a descartar o resultado da avaliação.
      
      De posse das informações acima, gostaríamos que você se pronunciasse acerca da avaliação.
      
      ( ) Dou meu consentimento para sua realização. 
      
      ( ) Não autorizo sua realização. 
      
      \begin{flushright}
      
      Brasília, \_\_\_\_\_ de maio de 2015. 
      
      \textbf{Participante:}  Nome: \_\_\_\_\_\_\_\_\_\_\_\_\_\_\_\_\_\_\_\_\_\_\_\_\_\_\_ 
      
      Assinatura:\_\_\_\_\_\_\_\_\_\_\_\_\_\_\_\_\_\_\_\_\_\_\_\_\_  
      
        \textbf{Avaliadores:}  Nome: \_\_\_\_\_\_\_\_\_\_\_\_\_\_\_\_\_\_\_\_\_\_\_\_\_\_\_ 
        
         Assinatura: \_\_\_\_\_\_\_\_\_\_\_\_\_\_\_\_\_\_\_\_\_\_\_\_\_ 
      
       \end{flushright}
  
  \chapter{Termo de consentimento assinado pelos usuários que participaram da avaliação}
  
    \section*{Usuário 1}
    \begin{figure}[!htbp]
      \centering
      \includegraphics[scale=0.7, angle=-90]{editaveis/figuras/termousuario1}
      \caption{Termo de consentimento assinado por um dos usuários que participaram da avaliação)}
      \label{termo_consentimento_1}
    \end{figure}
   
   \section*{Usuário 2}
    \begin{figure}[!htbp]
      \centering
      \includegraphics[scale=0.7, angle=-90]{editaveis/figuras/termousuario2}
      \caption{Termo de consentimento assinado por um dos usuários que participaram da avaliação)}
      \label{termo_consentimento_1}
    \end{figure}
	
  \chapter{Questionário ASQ (traduzido)}
      
      O seguinte questionário foi proposto por Lewis (\citeyear{lewis91}) e traduzido para o português pelos autores.
      
      \section*{Itens}
	
	\begin{enumerate}
	  \item No geral, estou satisfeito com a facilidade de completar essa tarefa.
	  \item No geral, estou satisfeito com a quantidade de tempo que levou para completar essa tarefa.
	  \item No geral, estou satisfeito com as informações de suporte (ajudas online, mensagens, documentação)
	    oferecidas ao se completar essa tarefa.
	\end{enumerate}
      
      \section*{Padrão de resposta}
      
	O ASQ segue um padrão de resposta gradativo de sete escalas. Quanto mais perto de 1, mais o usuário concorda com a 
	afirmação feita no respectivo item, enquanto quanto mais perto de 7, mais o usuário discorda com a afirmação feita no
	respectivo item. Ainda é possível escolher a opção de "Não aplicável".\\
            
	\noindent
	\textbf{CONCORDO FORTEMENTE}   1    2    3    4    5    6    7    \textbf{DISCORDO FORTEMENTE}
	
  \chapter{Questionário PSSUQ (traduzido)}
    
    O seguinte questionário foi proposto por Lewis (\citeyear{lewis02}) e traduzido para o português pelos autores.
      
      \section*{Itens}
	
	\begin{enumerate}
	  \item No geral, estou satisfeito com o quão fácil foi de utilizar esse sistema.

	  \item Foi simples utilizar esse sistema.

	  \item Eu pude efetivamente completar as tarefas e cenários utilizando esse sistema.

	  \item Eu fui capaz de completar as tarefas e cenários rapidamente utilizando esse sistema.

	  \item Eu fui capaz de completar as tarefas e cenários utilizando esse sistema, eficientemente.

	  \item Me senti confortável utilizando esse sistema.

	  \item Foi fácil aprender a utilizar o sistema.

	  \item Acredito que poderia me tornar produtivo rapidamente utilizando esse sistema.

	  \item O sistema forneceu mensagens de erros que diziam claramente como consertar os problemas.

	  \item Quando eu cometia erros utilizando o sistema, era possível recuperar facil e rapidamente.

	  \item As informações de suporte fornecidas pelo sistema eram claras.

	  \item Foi fácil achar a informação que eu queria.

	  \item As informações fornecidas pelo sistema eram fáceis de serem compreendidas.

	  \item A informação foi efetiva em me ajudar a completar as tarefas e cenários.

	  \item A organização da informação na tela do sistema era clara.

	  \item A interface desse sistema é agradável.

	  \item Gostei de utilizar a interface desse sistema, era interessante.

	  \item O sistema possui todas as funcionalidades e capacidades que eu esperava.

	  \item No geral, estou satisfeito com esse sistema.
	\end{enumerate}
      
      \section*{Padrão de resposta}
      
	O PSSUQ também segue um padrão de resposta gradativo de sete escalas. Quanto mais perto de 1, mais o usuário concorda com a 
	afirmação feita no respectivo item, enquanto quanto mais perto de 7, mais o usuário discorda com a afirmação feita no
	respectivo item. Ainda é possível escolher a opção de "Não aplicável".\\
            
	\noindent
	\textbf{CONCORDO FORTEMENTE}   1    2    3    4    5    6    7    \textbf{DISCORDO FORTEMENTE}
    
  
\end{anexosenv}

