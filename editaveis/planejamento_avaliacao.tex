\chapter{Planejamento da avaliação}

    A avaliação tem por objetivos avaliar a aceitação da tecnologia pelo usuário, validar ideias e designs alternativos e buscar 
    por problemas na interface e na interação. Essa seção aborda tópicos necessários para a realização da avaliação e, por fim, 
    apresenta o planejamento das avaliações feitas no projeto.
    
    \section{\textit{Framework} DECIDE}
    
    Avaliações bem planejadas seguem metas claras e perguntas adequadas Brasili \textit{et al.} (1994 \textit{apud} \cite{preece}).
    Para orientar as nossas avaliações, utilizaremos o \textit{framework} DECIDE.
    
    Este \textit{framework} nos oferece uma lista
    de checagem para orientar nas avaliações pessoas com pouca experiência, que é o caso da equipe de avaliadores. Para
    a orientação da avaliação do projeto foi utilizado o \textit{framework} apresentado por \citeauthor{preece} (\citeyear{preece}),
    que conta com os seguintes tópicos para a avaliação:
    
    \begin{itemize}
       \item \textbf{Determinar as metas que a avaliação irá abordar.}
       \subitem Segundo \cite{preece}, as metas escolhidas devem guiar a avaliação, ou seja, o resultado da avaliação deve propor melhorias 
       que atingem estas metas estabelecidas para que haja a evolução do protótipo. As metas que se deseja alcançar para a avaliação do 
       aplicativo \textit{MyPush} estão descritas na seção 4 deste documento.
    \end{itemize}
    
    \begin{itemize}
    
       \item \textbf{Explorar as questões específicas a serem respondidas.}
       \subitem Para que se alcance as metas de avaliação definidas, deve-se escolher questões cujas respostas satisfaçam à elas. 
        Em anexo, se encontram os questionários que foram definidos para que as metas no aplicativo \textit{MyPush} sejam testadas.
       
    \end{itemize}
    
    \begin{itemize}
    
       \item \textbf{Identificar questões práticas que devem ser abordadas.}
       \subitem Algumas questões práticas devem ser consideradas antes do início da avaliação. Deve ser feito uma seleção adequada dos usuários 
       que avaliarão a ferramenta, pois estes precisam representar a população pelo qual a aplicação será direcionada. Deve-se considerar quais 
       equipamentos a equipe irá utilizar para a avaliação: Câmeras, mesa, cadeira, caneta, papel, etc. Deve se considerar também se a equipe
       de avaliação possui conhecimento necessário para realizar a avaliação. Estes membros devem estar totalmente por dentrode todas as 
       funcionalidades que o protótipo possui. 
       
    \end{itemize}
       
    \begin{itemize}
    
       \item \textbf{Decidir como lidar com as questões éticas.}
       \subitem A equipe responsável pela avaliação da ferramenta deve se atentar sobre as questões éticas no processo da avaliação. Deve se manter a 
       privacidade das pessoas que farão parte da avaliação da ferramenta e isto deve ficar claro à eles antes da avaliação, quando são convidados a 
       participar da mesma através do termo de consentimento. Seus registros pessoais ficar ser confidenciais. Os usuários devem ser informados sobre 
       o tipo de registro das avaliações: filmagem, foto, etc.
       
    \end{itemize}
       
    \begin{itemize}
    
       \item \textbf{Avaliar, interpretar e apresentar os dados.}
       \subitem Os dados recolhidos nas avaliações do protótipo serão analisados, interpretados e, caso a equipe julgue necessário,
	considerados no desenvolvimento da aplicação.
   
   \end{itemize}
  
  \vfill
  \pagebreak
  \section{Estudos dos questionários}
    
    Os seguintes questionários foram analisados para a definição dos que seriam utilizados no projeto.
    
    \input{editaveis/estudo_questionarios}
    
  \section{Usuários para a avaliação}
      
    Nesta seção serão apresentados o perfil e a quantidade dos usuários selecionados para a avaliação.
    
    \subsection{Perfil dos usuários}
      
      Kuniavsky (2003 \textit{apud} \citeauthor{barbosa10}, \citeyear{barbosa10}) afirma que as melhores pessoas para
      se convidar para uma avaliação de IHC são as que precisarão do serviço oferecido pela aplicação.
      Testar a usabilidade de uma aplicação é tentar enxergar pelos olhos dos usuários, e por isso os participantes da avaliação 
      devem ser pertencentes ao público alvo da aplicação \cite{barbosa10}.
      
      A proposta do aplicativo é programar notificações sobre eventos, portanto o público alvo seria pessoas que costumam 
      participar de eventos diversos, como teatros, \textit{shows}, congressos, feiras, simpósios, entre outros.
      
    \subsection{Quantidade de usuários}
      
      Os autores na literatura divergem um pouco sobre a quantidade de usuários para uma avaliação, variando entre três e dez participantes.
      Nielsen (1993 \textit{apud} \citeauthor{barbosa10}, \citeyear{barbosa10}) propõe que sejam selecionados apenas cinco participantes,
      pois apresenta a melhor relação custo-benefício e permite que 85\% dos problemas sejam descobertos, enquanto para identificar mais
      5\% dos problemas seria necessário o dobro de usuários. 
      
      Seguindo a proposição de Nielsen, para as avaliações realizadas no projeto serão selecionados cinco usuários, devido a ser uma 
      quantidade suficiente para levantar os problemas de usabilidade e por não demandar muito tempo para as avaliações.
    
  \section{Termo de consentimento}
   
    O termo de consentimento é um termo a qual o usuário que será avaliado deve assinar antes da avaliação, 
    caso ele queira participar do processo de avaliação. Este termo assegura à equipe de avaliação que o usuário
    está ciente dos objetivos da avaliação e interessado em participar da mesma. O termo de consentimento utilizado
    para este trabalho segue o modelo feito por \citeauthor{termoconsentimento} (\citeyear{termoconsentimento}) e
    se encontra em anexo junto a dois termos preenchidos por usuários que participaram
    da primeira iteração de avaliação.
  
    \vfill
    \pagebreak
    \section{Planejamento das avaliações feitas no projeto}
      
      \subsection{Para o protótipo de papel}
      
	Essa seção apresenta o planejamento feito para as iterações de avaliação do protótipo de papel.
      
	\subsubsection{Iteração 1}
	  
	  \begin{itemize}
	   \item \textbf{Objetivo da avaliação}
	      
	      \subitem Essa iteração de avaliação tem por objetivo primário levantar requisitos para aplicação inicial.
	    
	   \item \textbf{Escopo da avaliação}
	      
	      \subitem 
		O usuário avaliará as funcionalidades existentes no protótipo de papel 
		(Programar notificação e gerenciar notificações), referentes aos requisitos funcionais iniciais levantados,
		para levantar possíveis requisitos.
	      
	   \item \textbf{Questões a se aplicar}
	      
	      \subitem Será aplicado apenas o questionário ASQ para avaliar e permitir o acompanhamento da evolução dos cenários.
	      
	   \item \textbf{Técnica de avaliação}
	      
	      \subitem 
		Serão utilizadas as técnicas de observação do usuário e de solicitação da opinião dos usuários
		sob o paradigma de avaliação \textit{'quick and dirty'}.
	      
	   \item \textbf{Questões práticas}
	      
	      \subitem Foi selecionado um grupo de 5 usuários com o perfil de uso da aplicação.
	      
	      \subitem Será necessário apenas o uso do protótipo de papel para a avaliação, não
		necessitando de outras ferramentas ou equipamentos.
	      
	   \item \textbf{Questões éticas}
	      
	      \subitem 
		Os usuários deverão assinar um termo de consentimento para participar da avaliação (em anexo).
	      
	  \end{itemize}
	
	\subsubsection{Iteração 2}
	  
	  \begin{itemize}
	  
	   \item \textbf{Objetivo da avaliação}
	      
	      \subitem Essa iteração de avaliação tem por objetivo primário levantar e estabilizar
		requisitos para aplicação inicial.
	    
	   \item \textbf{Escopo da avaliação}
	      
	      \subitem 
		O usuário avaliará as funcionalidades existentes no protótipo de papel 
		(Programar notificação e gerenciar notificações, com as novas mudanças do resultado da iteração 1),
		referentes aos requisitos funcionais levantados, para levantar possíveis requisitos.
	      
	   \item \textbf{Questões a se aplicar}
	      
	      \subitem Será aplicado apenas o questionário ASQ para avaliar e permitir o acompanhamento da evolução dos cenários.
	      
	   \item \textbf{Técnica de avaliação}
	      
	      \subitem 
		Serão utilizadas as técnicas de observação do usuário e de solicitação da opinião dos usuários
		sob o paradigma de avaliação \textit{'quick and dirty'}.
	      
	   \item \textbf{Questões práticas}
	      
	      \subitem Foi selecionado um grupo de 5 usuários com o perfil de uso da aplicação.
	      
	      \subitem Serão necessários o protótipo de papel e uma câmera para a realização da avaliação.
	      
	   \item \textbf{Questões éticas}
	      
	      \subitem 
		Os usuários deverão assinar um termo de consentimento para participar da avaliação (em anexo).
	   
	  \end{itemize}
      
      \subsection{Para o protótipo de alta fidelidade}
	
	Essa seção apresenta o planejamento feito para as iterações de avaliação do protótipo de alta fidelidade, feito na
	ferramenta \textit{Just in Mind}.
	
	\subsubsection{Iteração 1}
	
	  \begin{itemize}
	   \item \textbf{Objetivo da avaliação}
	      
	      \subitem Essa iteração de avaliação tem por objetivo primário avaliar os quesitos de usabilidade do protótipo.
	      
	      \subitem \textbf{Metas e princípios de usabilidade}\\
		O intuito da avaliação é analisar as seguintes metas para o protótipo:
		
		\subsubitem - Utilidade;
		\subsubitem - Eficácia e eficiência;
		\subsubitem - Interface esteticamente agradável.
		
	      Também será observado se a alguma heurística de usabilidade de Nielsen foi violada, para analisar possíveis 
	      falhas de usabilidade.
	    
	   \item \textbf{Escopo da avaliação}
	      
	      \subitem Essa iteração de avaliação avaliará os requisitos funcionais da aplicação levantados e incrementados
	       com o \textit{feedback} dos usuários na avaliação do protótipo de papel.
	      
	      \subitem \textbf{Cenários de avaliação com as tarefas a serem realizados pelos usuários}:
		
		\begin{itemize}
		  \item Cenário A - Criar uma nova notificação para determinado tema, sem adicionar um comentário.

		  \item Cenário B - Editar a hora da notificação criada e adicionar um comentário à notificação.
		    
		  \item Cenário C - Apagar a notificação criada.
		  
		\end{itemize}

	   \item \textbf{Questões a se aplicar}
	      
	      \subitem Será aplicado os questionários ASQ e o PSSUQ para avaliar e permitir o acompanhamento
		da evolução dos cenários e do sistema, no geral.
	      
	   \item \textbf{Técnica de avaliação}
	      
	      \subitem 
		Serão utilizadas as técnicas de observação do usuário e de solicitação da opinião dos usuários
		sob o paradigma de avaliação de teste de usabilidade.
	      
	   \item \textbf{Questões práticas}
	      
	      \subitem Foi selecionado um grupo de 5 usuários com o perfil de uso da aplicação.
	      
	      \subitem Será necessário o uso do protótipo em um \textit{smartphone} ou em um computador, e uma câmera para 
	      a recordação da avaliação.
	      
	   \item \textbf{Questões éticas}
	      
	      \subitem 
		Os usuários deverão assinar um termo de consentimento para participar da avaliação (em anexo).
	      
	  \end{itemize}
	  
      \subsubsection{Iteração 2}
	
	  \begin{itemize}
	   \item \textbf{Objetivo da avaliação}
	      
	      \subitem Essa iteração de avaliação tem por objetivo avaliar os quesitos de usabilidade do protótipo, considerando as
	      alterações feitas com os resultados da primeira iteração de avaliação.
	      
	      \subitem \textbf{Metas e princípios de usabilidade}\\
		O intuito da avaliação continua sendo analisar as seguintes metas para o protótipo:
		
		\subsubitem - Utilidade;
		\subsubitem - Eficácia e eficiência;
		\subsubitem - Interface esteticamente agradável.
		
	      Também será observado se a alguma heurística de usabilidade de Nielsen foi violada, para analisar possíveis 
	      falhas de usabilidade.
	    
	   \item \textbf{Escopo da avaliação}
	      
	      \subitem Essa iteração de avaliação avaliará os requisitos funcionais da aplicação levantados e incrementados
	       com o \textit{feedback} dos usuários na avaliação do protótipo de papel.
	      
	      \subitem \textbf{Cenários de avaliação com as tarefas a serem realizados pelos usuários}:
		
		\begin{itemize}
		  \item Cenário A - Criar uma nova notificação para determinado tema, sem adicionar um comentário.

		  \item Cenário B - Editar a hora da notificação criada e adicionar um comentário à notificação.
		    
		  \item Cenário C - Apagar a notificação criada.
		  
		\end{itemize}

	   \item \textbf{Questões a se aplicar}
	      
	      \subitem Será aplicado os questionários ASQ e o PSSUQ para avaliar e permitir o acompanhamento
		da evolução dos cenários e do sistema, no geral.
	      
	   \item \textbf{Técnica de avaliação}
	      
	      \subitem 
		Serão utilizadas as técnicas de observação do usuário e de solicitação da opinião dos usuários
		sob o paradigma de avaliação de teste de usabilidade.
	      
	   \item \textbf{Questões práticas}
	      
	      \subitem Foi selecionado um grupo de 5 usuários com o perfil de uso da aplicação.
	      
	      \subitem Será necessário o uso do protótipo em um \textit{smartphone} ou em um computador, e uma câmera para 
	      a recordação da avaliação.
	      
	   \item \textbf{Questões éticas}
	      
	      \subitem 
		Os usuários deverão assinar um termo de consentimento para participar da avaliação (em anexo).
	      
	  \end{itemize}
	  
	\subsubsection{Iteração 3}
	  
	  Com os resultados da primeira e segunda iteração de avaliação, foi possível medir a evolução do protótipo, e o resultado 
	  obtido pode ser visto com mais detalhes na seção 8.2.3. Com a análise dos resultados obtidos apresentados na seção 8.2.3,
	  foi possível ver que os cenários que apresentaram menor evolução foram os cenários B e C, o que demonstra que estes devem 
	  ser o foco dessa próxima iteração de avaliação.
	    
	  As dimensões do PSSUQ que apresentaram menor evolução foram as dimensões 
	  \textit{SysUse} e \textit{InfoQual}, que estão relacionadas com as metas de Utilidade, Eficácia e Eficiência. Com base 
	  nesses resultados, percebe-se que as metas que precisam de mais atenção nessa iteração de avaliação são as supracitadas.
	  
	  Com essas observações, segue abaixo o planejamento da terceira iteração de avaliação, que por restrições de tempo não poderá
	  ser executado no contexto deste projeto.
	  
	  \begin{itemize}
	   \item \textbf{Objetivo da avaliação}
	      
	      \subitem Essa iteração de avaliação tem por objetivo avaliar os quesitos de usabilidade do protótipo, considerando as
	      alterações feitas com os resultados da primeira e segunda iteração de avaliação.
	      
	      \subitem \textbf{Metas e princípios de usabilidade}\\
		O intuito da avaliação continua sendo analisar as seguintes metas para o protótipo:
		
		\subsubitem - Utilidade;
		\subsubitem - Eficácia e eficiência;
		
	      Também será observado se a alguma heurística de usabilidade de Nielsen foi violada, para analisar possíveis 
	      falhas de usabilidade.
	    
	   \item \textbf{Escopo da avaliação}
	      
	      \subitem Essa iteração de avaliação avaliará os seguintes cenários para medir a usabilidade do sistema:
	      
	      \subitem \textbf{Cenários de avaliação com as tarefas a serem realizados pelos usuários}:
		
		\begin{itemize}
		
		  \item Cenário B - Editar a hora da notificação criada e adicionar um comentário à notificação.
		    
		  \item Cenário C - Apagar a notificação criada.
		  
		\end{itemize}

	   \item \textbf{Questões a se aplicar}
	      
	      \subitem Será aplicado os questionários ASQ e o PSSUQ para avaliar e permitir o acompanhamento
		da evolução dos cenários e do sistema, no geral.
	      
	   \item \textbf{Técnica de avaliação}
	      
	      \subitem 
		Serão utilizadas as técnicas de observação do usuário e de solicitação da opinião dos usuários
		sob o paradigma de avaliação de teste de usabilidade.
	      
	   \item \textbf{Questões práticas}
	      
	      \subitem Foi selecionado um grupo de 5 usuários com o perfil de uso da aplicação.
	      
	      \subitem Será necessário o uso do protótipo em um \textit{smartphone} ou em um computador, e uma câmera para 
	      a recordação da avaliação.
	      
	   \item \textbf{Questões éticas}
	      
	      \subitem 
		Os usuários deverão assinar um termo de consentimento para participar da avaliação (em anexo).
	      
	  \end{itemize}
	
	