\chapter{Metas de usabilidade}

O sistema apresenta uma grande utilidade para o usuário, pois atendeu a uma necessidade que surgiu perante a situação. 
O sistema deve atender a essa necessidade de forma eficaz e eficiente, resolvendo o problema do usuário de forma rápida,
descomplicada e bem feita, de modo a facilitar também o aprendizado do uso. As funcionalidades do sistema devem estar visíveis 
para a facilidade do uso e o sistema deve sempre retornar um \textit{feedback} ao usuário a respeito das ações que estão sendo executadas.
O usuário deve ser capaz de controlar o sistema por completo, não podendo haver “becos sem saída”.

\chapter{Metas decorrentes da experiência do usuário}

Notificações de aplicativos tendem a ser algo indesejado e incomodam. Para que um aplicativo, cujo foco é enviar notificações,
não incomode neste aspecto, o aplicativo deve se manter interessante a ao mesmo tempo útil e prático. Algo com importância similar
é a aparência do aplicativo, precisando ser esteticamente agradável, de modo que seja compensador para o usuário quando ele abrir 
o aplicativo. 