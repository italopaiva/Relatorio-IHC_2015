\chapter{Metas para a aplicação}
  
  Essa seção apresenta as metas que se deseja alcançar com a aplicação.

  \section{Metas de usabilidade}
  
    “A usabilidade é geralmente considerada como o fator que assegura que os produtos são fáceis de usar, 
    eficientes e agradáveis - da perspectiva do usuário.” \cite{preece}.
    
    Segundo \citeauthor{preece} (\citeyear{preece}), a usabilidade está dividida nas seguintes metas:
    
    \begin{itemize}
       \item Ser eficaz no uso.
       \subitem Refere-se a o quanto o sistema é capaz de fazer o que ele realmente foi proposto a fazer.
    \end{itemize}
    
    \begin{itemize}
       \item Ser eficiente no uso.
       \subitem Refere-se a maneira com que o sistema presta assistência aos usuários na realização das tarefas.
    \end{itemize}
    
    \begin{itemize}
       \item Ser segura no uso.
       \subitem Refere-se a capacidade de o sistema de prestar proteção aos usuários para evitar que estes se encontrem 
       em situações perigosas e indesejáveis.
    \end{itemize}
    
    \begin{itemize}
       \item Ser de boa utilidade.
       \subitem Refere-se a o quão siginificativo o sistema pode ser no cotidiano dos usuários.
    \end{itemize}
    
    \begin{itemize}
       \item Ser fácil de aprender.
       \subitem Refere-se a facilidade que o usuário tem de aprender as funcionalidades do sistema.
    \end{itemize}
    
    \begin{itemize}
       \item Ser fácil de se lembrar como se usa.
       \subitem Referre-se a facilidade que o usuário tem de se lembrar de como se usa as funcionalidades do sistema.
    \end{itemize}
  
    O sistema apresenta uma grande utilidade para o usuário, pois atendeu a uma necessidade que surgiu perante a situação. 
    O sistema deve atender a essa necessidade de forma eficaz e eficiente, resolvendo o problema do usuário de forma rápida,
    descomplicada e bem feita, de modo a facilitar também o aprendizado do uso. As funcionalidades do sistema devem estar visíveis 
    para a facilidade do uso e o sistema deve sempre retornar um \textit{feedback} ao usuário a respeito das ações que estão sendo executadas.
    O usuário deve ser capaz de controlar o sistema por completo, não podendo haver “becos sem saída”.

  \section{Metas decorrentes da experiência do usuário}
    
    “O objetivo de desenvolver produtos interativos agradáveis, divertidos, esteticamente apreciáveis, etc. 
    está principalmente na experiência que estes proporcionarão ao usuário, isto é, como o usuário se 
    sentirá na interação com o sistema.” \cite{preece}.
    
    Segundo \citeauthor{preece} (\citeyear{preece}), o design de interação das aplicações estão cada vez mais preocupadas em seguirem 
    metas decorrentes da experiência do usuário, que são:
    
    \begin{itemize}
       \item Satisfatórios.
       \subitem Refere-se a capacidade que o sistema tem de satisfazer as necessidades do usuário.
    \end{itemize}
    
    \begin{itemize}
       \item Agradáveis.
       \subitem Refere-se a capacidade que o sistema tem de agradar ao usuário.
    \end{itemize}
    
    \begin{itemize}
       \item Divertidos.
       \subitem Refere-se ao nível de divertimento que o sistema pode causar ao usuário.
    \end{itemize}
    
    \begin{itemize}
       \item Interessantes.
       \subitem Refere-se a capacidade que o sistema tem de despertar interesse por parte do usuário em utilizá-lo.
    \end{itemize}
    
    \begin{itemize}
       \item úteis.
       \subitem Refere-se ao nível de utilidade do sistema no cotidiano do usuário.
    \end{itemize}
    
    \begin{itemize}
       \item Motivadores.
       \subitem Refere-se ao nível de motivação que o sistema pode causar ao usuário.
    \end{itemize}
    
    \begin{itemize}
       \item esteticamente apreciáveis.
       \subitem Refere-se ao quão agradável o \textit{design} do sistema pode ser ao usuário.
    \end{itemize}
    
    \begin{itemize}
       \item Incentivadores de criatividade.
       \subitem Refere-se ao nível de criatividade que o sistema pode despertar no usuário.
    \end{itemize}
    
    \begin{itemize}
       \item Compensadores.
       \subitem Refere-se ao quao compensador o uso do sistema pode ser ao usuário.
    \end{itemize}
    
    \begin{itemize}
       \item Emocionalmente adequados.
       \subitem Refere-se aos tipos de emoções que o sistema causa ao usuário e se estas estão adequadas ao sobjetivos do sistema.
    \end{itemize}
    
    Notificações de aplicativos tendem a ser algo indesejado e incomodam. Para que um aplicativo, cujo foco é enviar notificações,
    não incomode neste aspecto, o aplicativo deve se manter interessante a ao mesmo tempo útil e prático. Algo com importância similar
    é a aparência do aplicativo, precisando ser esteticamente agradável, de modo que seja compensador para o usuário quando ele abrir 
    o aplicativo.
    