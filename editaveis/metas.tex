\chapter{Metas e princípios de usabilidade}
  
  Essa seção apresenta as metas que se deseja alcançar com a aplicação.

  \section{Metas de usabilidade}
  
    “A usabilidade é geralmente considerada como o fator que assegura que os produtos são fáceis de usar, 
    eficientes e agradáveis - da perspectiva do usuário.” \cite{preece}.
    
    Segundo \citeauthor{preece} (\citeyear{preece}), a usabilidade pode ser dividida nas seguintes metas:
    
    \begin{itemize}
       \item Ser eficaz no uso.
       \subitem Refere-se a o quanto o sistema é capaz de fazer o que ele realmente foi proposto a fazer.
 
       \item Ser eficiente no uso.
       \subitem Refere-se a maneira com que o sistema presta assistência aos usuários na realização das tarefas.

       \item Ser segura no uso.
       \subitem Refere-se a capacidade de o sistema de prestar proteção aos usuários para evitar que estes se encontrem 
       em situações perigosas e indesejáveis.

       \item Ser de boa utilidade.
       \subitem Refere-se a o quão siginificativo o sistema pode ser no cotidiano dos usuários.

       \item Ser fácil de aprender.
       \subitem Refere-se a facilidade que o usuário tem de aprender as funcionalidades do sistema.

       \item Ser fácil de se lembrar como se usa.
       \subitem Refere-se a facilidade que o usuário tem de se lembrar de como se usa as funcionalidades do sistema.
    \end{itemize}
    
  \section{Princípios de usabilidade}
  
    Outra forma de classificar a usabilidade se dá por meio de abstrações generalizáveis utilizadas para orientar os \textit{designers}
    sobre aspectos diferentes do \textit{design} no desenvolvimento, os princípios de usabilidade \cite{preece}. Os mais comuns são:
    
    \begin{itemize}
     \item Visibilidade
	
	\subitem Quanto mais visíveis as funcionalidades do sistema, mais fácil será para o usuário saber como proceder \cite{preece}.
      
     \item \textit{Feedback}
	
	\subitem Relacionado ao princípio da Visibilidade, o \textit{feedback} diz respeito ao retorno de informações sobre as
	  ações realizadas pelo o usuário, permitindo a continuidade da operação \cite{preece}.
     
     \item Restrições
     
	\subitem As restrições está intrisecamente relacionadas com a prevenção de erros pelo usuário, impedindo-o de realizar 
	  tarefas que não são possíveis em determinado contexto \cite{preece}.

     \item Mapeamento
	
	\subitem Refere-se ao mapeamento dos controles e os seus efeitos no mundo real \cite{preece}.
	  Os controles de um determinado conjunto de funções devem manter a consistência com o mundo real para que haja uma
	  percepção natural por parte do usuário.
	
     \item Consistência
	
	\subitem A consistência está relacionada com a obediências às regras estabelecidas para as operações do sistema. 
	  Uma interface consistente não permite exceções às regras, mantendo o mesmo padrão para realizar operações
	  semelhantes \cite{preece}.
	
     \item \textit{Affordance}
	
	\subitem Segundo Norman (1998, \textit{apud} \citeauthor{preece}, \citeyear{preece}) \textit{affordance} é "dar uma pista".
	  Está relacionado ao atributo de algum objeto que permite às pessoas inferir como utilizá-lo \cite{preece}.
	  
    \end{itemize}
   
  \section{Metas decorrentes da experiência do usuário}
    
    “O objetivo de desenvolver produtos interativos agradáveis, divertidos, esteticamente apreciáveis, etc. 
    está principalmente na experiência que estes proporcionarão ao usuário, isto é, como o usuário se 
    sentirá na interação com o sistema.” \cite{preece}.
    
    Segundo \citeauthor{preece} (\citeyear{preece}), o design de interação das aplicações estão cada vez mais preocupadas em seguirem 
    metas decorrentes da experiência do usuário, que são:
    
    \begin{itemize}
       \item Satisfatórios.
       \subitem Refere-se a capacidade que o sistema tem de satisfazer as necessidades do usuário.

       \item Agradáveis.
       \subitem Refere-se a capacidade que o sistema tem de agradar ao usuário.

       \item Divertidos.
       \subitem Refere-se ao nível de divertimento que o sistema pode causar ao usuário.

       \item Interessantes.
       \subitem Refere-se a capacidade que o sistema tem de despertar interesse por parte do usuário em utilizá-lo.

       \item Úteis.
       \subitem Refere-se ao nível de utilidade do sistema no cotidiano do usuário.

       \item Motivadores.
       \subitem Refere-se ao nível de motivação que o sistema pode causar ao usuário.

       \item Esteticamente apreciáveis.
       \subitem Refere-se ao quão agradável o \textit{design} do sistema pode ser ao usuário.

       \item Incentivadores de criatividade.
       \subitem Refere-se ao nível de criatividade que o sistema pode despertar no usuário.

       \item Compensadores.
       \subitem Refere-se ao quão compensador o uso do sistema pode ser ao usuário.

       \item Emocionalmente adequados.
       \subitem Refere-se aos tipos de emoções que o sistema causa ao usuário e se estas estão adequadas ao sobjetivos do sistema.
    \end{itemize}
    
    \section{Metas e princípios para a aplicação}
    
      Para a aplicação proposta, o sistema deve apresentar uma grande utilidade para o usuário, pois precisa atender uma necessidade
      que possa surgir em determinada situação. O sistema deve atender a essa necessidade de forma eficaz, resolvendo o problema
      do usuário de forma descomplicada e bem feita, de modo a facilitar também o aprendizado do uso. As funcionalidades do sistema
      devem estar visíveis para a facilidade do uso e o sistema deve sempre retornar um \textit{feedback} ao usuário a respeito
      das ações que estão sendo executadas.
      
      Portanto, as metas e princípios de usabilidade desejadas para a aplicação são:
      
      \begin{itemize}
	\item Utilidade;
	\item Eficácia;
	\item Eficiência;
	\item Visibilidade;
	\item \textit{Feedback}.
      \end{itemize}
    
      Notificações de aplicativos tendem a ser algo indesejado e incomodam. Para que um aplicativo cujo foco é enviar notificações
      não incomode neste aspecto, o aplicativo deve se manter interessante a ao mesmo tempo útil e prático. Algo com importância similar
      é a aparência do aplicativo, precisando ser esteticamente agradável.
      
      Portanto, espera-se que a aplicação, para o usuário, seja:
      
      \begin{itemize}
       \item Interessante;
       \item Útil;
       \item Esteticamente agradável.
     \end{itemize}

      
    